%!TEX root=../document.tex

\section{Einführung}
Dieses Protokoll beschreibt meine Vorgänge bei der Übung DezSys-L01, welche die Verwendung von Namensdiensten in JAVA und auch OS-Ebene(Linux) vertieft.
\subsection{Ziele}
Das Ziel dieser Übung ist die Funktionsweise eines Namens/verzeichnisdiestes zu verstehen und Erfahrung mit der Administration zu sammeln. Ebenso soll die Verwendung des Dienstes aus einer Anwendung heraus mit Hilfe der JNDI geübt werden und auch ein CLI zur Administration angewendet werden.



\subsection{Voraussetzungen}

\begin{itemize}
	\item Grundlagen Namensdienst/Verzeichnisdienst
	\item Administration eines LDAP Dienstes
	\item Verwendung von Commandline Werkzeugen fuer LDAP (LDAPSEARCH)
	\item Grundlagen der JNDI API für eine JAVA Implementierung
	\item Verwendung einer virtuellen Instanz für den Betrieb des Verzeichnisdienstes
\end{itemize}




\subsection{Aufgabenstellung}
\begin{itemize}
	\item In der zurverfügung gestellten VM den LDAP Service benutzen um 2 Poxis Groups und 5 User Accounts zu erstellen.
	\item Es soll eine Java Applikation zur Anbindung an dass LADP-Verzeichnis entwickelt werden und es soll überprüft werden ob es einen Eintrag 	cn=max.mustermann,dc=nodomain,dc=com gibt und was der GivenName dieses Eintrags ist..
	\item Mit der LINUX CLI LDAPSEARCH sollen 3 Suchanfragen durchgeführt werden.
\end{itemize}
Nun kommt ein Seitenumbruch, um eine klare Trennung der Schülerarbeit zu bestimmen.
\clearpage
